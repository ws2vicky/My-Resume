% !TEX TS-program = xelatex
% !TEX encoding = UTF-8 Unicode
% !Mode:: "TeX:UTF-8"

\documentclass{resume}
\usepackage{zh_CN-Adobefonts_external} % Simplified Chinese Support using external fonts (./fonts/zh_CN-Adobe/)
% \usepackage{NotoSansSC_external}
% \usepackage{NotoSerifCJKsc_external}
% \usepackage{zh_CN-Adobefonts_internal} % Simplified Chinese Support using system fonts
\usepackage{linespacing_fix} % disable extra space before next section
\usepackage{cite}

\begin{document}
\pagenumbering{gobble} % suppress displaying page number

\name{王杉}
\contactInfo{前端开发}{男}{1年}{湖南长沙}

\basicInfo{
  \github[https://github.com/ws2vicky]{ https://github.com/ws2vicky}\textperiodcentered\
  \email{3679657098@qq.com} \textperiodcentered\
  \phone{18094504568} \textperiodcentered\
  % \linkedin[billryan8]{https://www.linkedin.com/in/billryan8}
  \WeChat{eihxjd}
}

\section{\faGraduationCap\  教育背景}
\datedsubsection{\textbf{廊坊师范学院}, 河北, 廊坊, 全日制}{2023 年毕业}
\textit{本科}\ 软件工程
% \datedsubsection{\textbf{西安电子科技大学}, 西安, 陕西}{2009 -- 2013}
% \textit{学士}\ 通信工程


\section{\faCogs\ 技能}
% increase linespacing [parsep=0.5ex]
\begin{itemize}[parsep=0.5ex]
  \item 熟悉使用 Html、Css、JavaScript、es6、flex 等布局
  \item 熟悉前端主流技术 Vue2、Vue3、Uni-app 、微信小程序主流技术并有实战经验
  \item 熟练使用 Element、Ant-designUI 框架,Sass、Less 等 CSS 预处理器,高效完成界面效果
  \item 熟悉使用 Git、Github、Gitlab 等版本管理工具,熟练使用 Git 命令
  \item 熟悉 Webpack、Vite 前端项目构建工具
  \item 能够独立通过 devtools 和控制台 network、console、debugger 解决问题,有强烈的代码洁癖
  \item 熟悉使用组件化开发模式,提高开发效率和代码复用率

\end{itemize}

\section{\faUsers\ 工作经历}
\datedsubsection{\textbf{重庆软微科技有限公司} ,重庆,渝北}{2023年4月-2024年3月}
\role{职位}{前端开发}
\begin{itemize}
  \item 负责对接合作公司产品的前端工作,以及数据类型展示
  \item 负责公司产品后台和 Web、app 前端开发工作
  \item 与后端同事沟通确定技术规范完成接口测试,根据产品反馈的业务需求进行功能需求修改
  \item 封装各类组件及 api 接口提供系统的可维护性、可读性,提高产品用户体验及交互效果
\end{itemize}

\section{\faUsers\ 项目经历}

\datedsubsection{\textbf{分布式科学上网姿势}}{2014年6月 -- 至今}
\role{Golang, Linux}{个人项目,和富帅糕合作开发}
\begin{onehalfspacing}
  分布式负载均衡科学上网姿势, https://github.com/cyfdecyf/cow
  \begin{itemize}
    \item 修复了连接未正常关闭导致文件描述符耗尽的 bug
    \item 使用Chord 哈希 URL, 实现稳定可靠地分流
    \item xxx (尽量使用量化的客观结果)
  \end{itemize}
\end{onehalfspacing}

\datedsubsection{\textbf{\LaTeX\ 简历模板}}{2015 年5月 -- 至今}
\role{\LaTeX, Python}{个人项目}
\begin{onehalfspacing}
  优雅的 \LaTeX\ 简历模板, https://github.com/billryan/resume
  \begin{itemize}
    \item 容易定制和扩展
    \item 完善的 Unicode 字体支持,使用 \XeLaTeX\ 编译
    \item 支持 FontAwesome 4.5.0
  \end{itemize}
\end{onehalfspacing}

% Reference Test
%\datedsubsection{\textbf{Paper Title\cite{zaharia2012resilient}}}{May. 2015}
%An xxx optimized for xxx\cite{verma2015large}
%\begin{itemize}
%  \item main contribution
%\end{itemize}

\section{\faCogs\ IT 技能}
% increase linespacing [parsep=0.5ex]
\begin{itemize}[parsep=0.5ex]
  \item 编程语言: C == Python > C++ > Java
  \item 平台: Linux
  \item 开发: xxx
\end{itemize}

\section{\faHeartO\ 获奖情况}
\datedline{\textit{第一名}, xxx 比赛}{2013 年6 月}
\datedline{其他奖项}{2015}

\section{\faInfo\ 其他}
% increase linespacing [parsep=0.5ex]
\begin{itemize}[parsep=0.5ex]
  \item 技术博客: http://blog.yours.me
  \item GitHub: https://github.com/username
  \item 语言: 英语 - 熟练(TOEFL xxx)
\end{itemize}

%% Reference
%\newpage
%\bibliographystyle{IEEETran}
%\bibliography{mycite}
\end{document}
